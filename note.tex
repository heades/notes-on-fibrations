\documentclass{article}

\usepackage{amssymb,amsmath,mathtools,amsthm}
\usepackage{cmll}
\usepackage{stmaryrd}
\usepackage{mathpartir}
\usepackage{supertabular}
\usepackage{color}
\usepackage{fullpage}
\usepackage{verbatim}
\usepackage[textwidth=2cm]{todonotes}
\usepackage{enumitem}
\usepackage{hyperref}
\usepackage{mnsymbol}

\newenvironment{changemargin}[2]{%
  \begin{list}{}{%
      \setlength{\topsep}{0pt}%
      \setlength{\leftmargin}{#1}%
      \setlength{\rightmargin}{#2}%
      \setlength{\listparindent}{\parindent}%
      \setlength{\itemindent}{\parindent}%
      \setlength{\parsep}{\parskip}%
    }%
  \item[]}{\end{list}}

\newtheorem{theorem}{Theorem}[section]
\newtheorem{corollary}{Corollary}[theorem]
\newtheorem{lemma}[theorem]{Lemma}
\newtheorem{definition}[theorem]{Definition}

\begin{document}

\title{Notes on Fibrational Semantics of Simple, Polymorphic, and Dependent Type Theory}
\author{Harley Eades III}
\maketitle

\section{The Simple Fibration}
\label{sec:the_simple_fibration}

\begin{definition}
  \label{def:CT-structure}
  A \emph{CT-structure} is a pair $(\mathbb{B},T)$ where $\mathbb{B}$
  is a category with finite products, and $T \subseteq
  \mathsf{Obj}(\mathbb{B})$ is a collection of types.
\end{definition}

A CT-structure $(\mathbb{B},T)$ should be thought of as a category of
contexts $\mathbb{B}$ whose types draw their atomic elements from $T$.
Given contexts $\Gamma,\Delta \in \mathsf{Obj}(\mathbb{B})$, their concatenation
$(\Gamma,\Delta) = (\Gamma \times \Delta)$.

\begin{definition}
  \label{def:simple-total-cat}
  ...
\end{definition}


\begin{definition}
  \label{def:simple-fibration}
  ...
\end{definition}


% section the_simple_fibration (end)


\appendix

\input{appendix}

\end{document}

%%% Local Variables:
%%% mode: latex
%%% TeX-master: t
%%% End:
