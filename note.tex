\documentclass{article}

\usepackage{amssymb,amsmath,mathtools,amsthm}
\usepackage{cmll}
\usepackage{stmaryrd}
\usepackage{mathpartir}
\usepackage{supertabular}
\usepackage{color}
\usepackage{fullpage}
\usepackage{verbatim}
\usepackage[textwidth=2cm]{todonotes}
\usepackage{enumitem}
\usepackage{hyperref}
\usepackage{mnsymbol}

\newenvironment{changemargin}[2]{%
  \begin{list}{}{%
      \setlength{\topsep}{0pt}%
      \setlength{\leftmargin}{#1}%
      \setlength{\rightmargin}{#2}%
      \setlength{\listparindent}{\parindent}%
      \setlength{\itemindent}{\parindent}%
      \setlength{\parsep}{\parskip}%
    }%
  \item[]}{\end{list}}

\newtheorem{theorem}{Theorem}[section]
\newtheorem{corollary}{Corollary}[theorem]
\newtheorem{lemma}[theorem]{Lemma}
\newtheorem{definition}[theorem]{Definition}

\begin{document}

\title{Notes on Fibrational Semantics of Simple, Polymorphic, and Dependent Type Theory}
\author{Harley Eades III}
\maketitle

\section{The Simple Fibration}
\label{sec:the_simple_fibration}
...

% section the_simple_fibration (end)


\appendix

\input{appendix}

\end{document}

%%% Local Variables:
%%% mode: latex
%%% TeX-master: t
%%% End:
